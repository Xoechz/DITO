\chapter{Kurzfassung}
\label{chap:kurzfassung}

\begin{german}
    Den Überblick über verteilte Softwaresysteme zu behalten, ist bereits eine Herausforderung.
    Wenn jedoch große Datenmengen zwischen mehreren Endpunkten übertragen werden,
    scheitert traditionelles Logging oft daran, Fragen zur Datenqualität und Zuverlässigkeit 
    der Systeme zu beantworten.
    
    Traces und Metriken können helfen, diese Probleme zu bewältigen.
    Die Programme, die diese Daten übertragen, sind jedoch häufig als
    unabhängige Hintergrundprozesse mit unterschiedlichen Ausführungszeiten
    implementiert, oft sogar auf voneinander unabhängigen Maschinen.
    
    Glücklicherweise ermöglicht OpenTelemetry die Vereinheitlichung von Traces über mehrere Maschinen
    hinweg mithilfe eines Collectors.
    Dennoch bleibt das Problem des Trace-Scopes bestehen:
    Ein Trace umfasst in der Regel eine Anfrage oder die Ausführung eines Jobs,
    was es schwierig macht, einzelne Datensätze nachzuverfolgen.
    
    Der Fokus dieser Arbeit liegt darauf, dieses Problem zu lösen.
    Ziel ist es, eine Möglichkeit zu schaffen, Traces zu sammeln, zu vereinheitlichen
    und anschließend in neue Traces aufzuteilen, die den Weg einzelner Datenpunkte durch das System zeigen.
    Die Arbeit umfasst die Erstellung von Traces mit semantischen Ereignissen auf
    Job-Ebene, die Verknüpfung von Traces verschiedener Jobs, die an denselben Daten arbeiten,
    das Aufteilen und Abtasten dieser Traces in jeweils einen Trace pro Datenpunkt in einem OpenTelemetry Collector
    sowie die Generierung von Metriken aus diesen Traces, um ein Gesamtbild des Datentransfers zu erhalten.
\end{german}
