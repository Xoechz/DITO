\chapter{Kurzfassung}

\begin{german}
    Den Überblick über verteilte Softwaresysteme zu behalten, ist schon schwierig genug, 
    aber wenn große Datenmengen von und zu mehreren Endpunkten transportiert werden,
    scheitert traditionelles Logging oft daran Fragen über die Datenqualität
    und Programmzuverlässigkeit zu beantworten.
    
    Traces und Metriken können dabei helfen, diese Probleme zu bewältigen,
    aber die Programme die diese Daten transferieren, sind oft als
    unabhängige Hintergrund Prozesse mit unterschiedlichen Ausführungszeiten
    implementiert, oft sogar auf unabhängigen Maschinen.
    
    Glücklicherweise kann man mit OpenTelemetry Traces über mehrere Maschinen 
    hinweg durch einen Kollektor vereinheitlichen.
    Dadurch bleibt aber noch das Problem des Umfangs eines Traces.
    
    Der Umfang eines Traces ist normalerweise eine Anfrage oder Jobausführung.
    Das macht es schwierig den Überblick über einzelne Datensätze zu behalten.
    Der Fokus dieser Arbeit liegt darauf dieses Problem zu lösen.\par
    
    Das Ziel ist es, eine Möglichkeit zu schaffen, Traces zu sammeln, zu vereinheitlichen
    und dann in neue Traces aufzuteilen, die  zeigen, wie sich Datenpunkte durch das System
    bewegen. 
    
    Die Arbeit erstreckt sich von der Erstellung von Traces mit semantischen Ereignissen auf der Job Ebene,
    über die Verbindung von Traces von verschieden Jobs, die an den gleichen Daten arbeiten,
    dann das Aufteilen und Abtasten der Traces in je einen Trace für jeden Datenpunkt in einem OpenTelemetry
    Kollektor und schließlich die Generierung von Metriken aus diesen Traces, um ein Gesamtbild
    des Datentransfers zu erhalten.
\end{german}
